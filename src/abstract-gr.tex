\chapter*{\abstractname}
\addcontentsline{toc}{chapter}{\abstractname}

Το \en{cloud computing} είναι η κυρίαρχη προσέγγιση προς την υπολογιστική
υποδομή, βασιζόμενο στην τεχνολογία της εικονικοποίησης (\en{virtualization}).
Καθώς το \en{cloud} επεκτείνεται, καθίσταται επιτακτική η αποδοτική χρήση των
υπολογιστικών του πόρων από το λογισμικό. Προς αυτό τον σκοπό, μια λύση είναι τα
\emph{\en{unikernels}}, πυρήνες λειτουργικών συστημάτων εξειδικευμένοι για να
τρέχουν μία μόνο εφαρμογή, εξοικονομώντας πόρους σε σχέση με έναν γενικού σκοπού
πυρήνα. Η αποδοτική πρόσβαση των \en{virtualized guests} στους πόρους του
\host{} συστήματος είναι μια μεγάλη πρόκληση για την εικονικοποίηση. Σε αυτόν
τον τομέα, σημαντική συνεισφορά αποτελεί το \emph{\en{virtio}}, μια προδιαγραφή
\en{paravirtualized} συσκευών για την αποδοτική χρήση των πόρων του \host{}.
Για το διαμοιρασμό αρχείων ανάμεσα σε \host{} και \guest{} έχει προταθεί το
\emph{\viofs{}}, μια \en{virtio} συσκευή που προσφέρει στον \guest{} πρόσβαση σε
έναν κατάλογο του συστήματος αρχείων του \host{}, παρέχοντας υψηλές επιδόσεις
και σημασιολογία τοπικού συστήματος αρχείων.

Αντικείμενο της παρούσας διπλωματικής εργασίας είναι η υλοποίηση και αξιολόγηση
της χρήσης του \viofs{} σε ένα \en{unikernel}, συγκεκριμένα στο \osv{}.
Δείχνουμε ότι ο συνδυασμός αυτών των δύο έχει σημαντικά πλεονεκτήματα, τόσο από
άποψη επιδόσεων, όπου επιτυγχάνονται αποτελέσματα συγκρίσιμα με τα τοπικά
συστήματα αρχείων, όσο και από διαχειριστική άποψη στο πλαίσιο του \en{cloud}.
Επίσης, τα παραπάνω γίνονται σε πλήρη ενσωμάτωση με το έργο ανοιχτού λογισμικού
του \en{unikernel} στο οποίο βασιστήκαμε. Έτσι, το προϊόν της εργασίας αποκτά
πρακτική αξία, καθώς αποτελεί χρήσιμη συνεισφορά στο έργο, επιτυγχάνοντας έτσι
έναν κεντρικό, μη τεχνικό, στόχο της. Ταυτόχρονα, εξερευνούμε τον τρόπο
λειτουργίας των έργων ανοιχτού λογισμικού και των κοινοτήτων που σχηματίζονται
γύρω τους, καθώς γινόμαστε ενεργά μέλη μίας από αυτές.

\section*{Λέξεις κλειδιά}
\noindent
εικονικοποίηση, νέφος, σύστημα αρχείων, \en{unikernel}, \en{virtio}, \osv{},
\viofs{}, \qemu{}
