\chapter{Συμπεράσματα και μελλοντικές επεκτάσεις}

Στο πλαίσιο της παρούσας εργασίας επεκτείναμε επιτυχώς την υλοποίηση του
\viofs{} στο \osv{} ώστε να μπορεί να χρησιμοποιήσει το \en{DAX window} (μόνο
για ανάγνωση) και να εκκινήσει με \viofs{} \en{root file system}. Για το πρώτο
χρειάστηκε να προσθέσουμε έναν διαχειριστή (\en{manager}) για το \en{DAX
window}, προκειμένου να επιτύχουμε αποδοτική χρήση του υπό την παρουσία
πολλαπλών απεικονίσεων. Επίσης, αξιολογήσαμε την υλοποίηση μας σε διαφορετικά
σενάρια, συμπεραίνοντας ότι βελτιώνει πολύ σημαντικά τις επιδόσεις του \viofs{},
οι οποίες στις περισσότερες περιπτώσεις καθίστανται με αυτό ανταγωνιστικές των
αποκλειστικών συστημάτων αρχείων στο \osv{} και πάντα καλύτερες από του
\en{NFS}, του μόνου άλλου διαθέσιμου κοινόχρηστου συστήματος αρχείων σε αυτό.
Τέλος, είδαμε στην πράξη την αξία του κοινόχρηστου συστήματος αρχείων, με το εκ
νέου χτίσιμο του \en{unikernel} να γίνεται περιττό στις περισσότερες περιπτώσεις
αλλαγών στην εφαρμογή.

Η εμπειρία χρήσης και ανάπτυξης στο \osv{} ήταν πολύ ενδιαφέρουσα, με κάποια
μεγάλα πλεονεκτήματα και κάποια σημεία που μπορούν να βελτιωθούν. Το κυριότερο
πλεονέκτημα απορρέει από τη χρήση της \en{C++}, σε συνδυασμό με τη φύση του
\en{library OS} και τη σημαντική δουλειά που έχει γίνει ήδη στον πυρήνα του
\osv{}: συχνά κατά την ανάπτυξη έχει κανείς την εντύπωση ότι γράφει κώδικα
εφαρμογής και όχι κώδικα συστήματος, με παροχές όπως δυναμική και ημιαυτόματη
(μέσω \en{smart pointers}) διαχείριση μνήμης και πρότυπες βιβλιοθήκες
διαθέσιμες, όπως στις συνήθεις εφαρμογές. Ο ιδιόμορφος κύκλος ανάπτυξης επίσης
συγκρίνεται περισσότερο με αυτόν μίας εφαρμογής σε γενικού σκοπού λειτουργικό,
χάρη στους σύντομους χρόνους χτισίματος και εκκίνησης. Το \en{debugging}
αποτελεί παραδοσιακά πρόκληση στα \en{unikernels}, όμως στην εμπειρία μας, το
\osv{} παρέχει πολύτιμη υποστήριξη γι' αυτό \cite{osv-wiki:debugging} και βέβαια
η σχετική ευκολία προγραμματισμού βοηθάει στο να αποφευχθούν λάθη εξ' αρχής. Η
τεκμηρίωση είναι ένας τομέας που σαφώς θα μπορούσε να είναι καλύτερος, μιας και
πολύ συχνά ο μόνος τρόπος να καταλάβει κανείς πως λειτουργεί κάτι είναι να
διαβάσει τον αντίστοιχο κώδικα, κάτι που δύναται να αποθαρρύνει δυνητικούς νέους
χρήστες. Αυτό φυσικά είναι ένα συχνό πρόβλημα σε έργα λογισμικού με πολύ λίγους
συνεισφέροντες (\en{contributors}) και αρκεί συνειδητή προσπάθεια για να
αλλάξει.

% TODO OPT: Unikernels in general (preference and forecast): ecosystem
% fragmentation, high-level language, ABI compatibility, target platform (light
% hypervisors), tooling

Στην μη-τεχνική πτυχή της εργασίας, είχαμε την ευκαιρία να συνεισφέρουμε
χρήσιμες προσθήκες σε ένα έργο ελεύθερου λογισμικού, ενώ ήρθαμε σε επαφή και
γίναμε μέρος της κοινότητας δύο τέτοιων έργων, συμμετείχαμε στις διαδικασίες
και γνωρίσαμε τις ροές εργασίας (\en{workflows}) τους. Έτσι επιτύχαμε έναν εξ'
αρχής στόχο, αυτόν της ανταπόδοσης στην κοινότητα του έργου στο οποίο
βασιστήκαμε.

Αμφότερες οι κοινότητες με τις οποίες ήρθαμε σε επαφή, αν και διαφορετικές σε
μέγεθος και ενασχόληση των περισσότερων μελών τους (μικρή και καθαρά εθελοντική
στο \osv{}, μεγαλύτερη και επαγγελματική στο \viofs{}) είχαν βασικά κοινά
στοιχεία. Αυτά δεν ήταν άλλα από την ανοιχτότητα, την υποδοχή, ενθάρρυνση και
υποστήριξη νέων μελών και τη διαθεσιμότητα και προθυμία για συλλογική ανάπτυξη
μέσω συζήτησης, σχολιασμού, καλοπροαίρετης κριτικής και επιβράβευσης.
% TODO OPT: Point to discussions on the mailing lists?

Όσον αφορά επεκτάσεις της δουλειάς μας στο \osv{}, μία που κρίνεται ιδιαίτερα
ενδιαφέρουσα είναι η απεικόνιση των αρχείων στη μνήμη της εφαρμογής μέσω
\texttt{\en{mmap()}}. Η πρόκληση εδώ είναι να δοθεί ουσιαστικά πρόσβαση στο
\en{DAX window} διατηρώντας τη σημασιολογία της \texttt{\en{mmap()}},
εξαλείφοντας έτσι την αντιγραφή που γίνεται αναπόφευκτα από το \en{DAX window}
προς το \en{buffer} της εφαρμογής όταν χρησιμοποιείται η \texttt{\en{read()}}. Η
μεγαλύτερη δυνατή επέκταση βεβαίως θα ήταν η υποστήριξη εγγραφών, με ή χωρίς το
\en{DAX window}, ώστε το σύστημα αρχείων αν είναι πλέον \en{read-write}.
Εκτιμάται ότι αυτό θα απαιτούσε σημαντική προσπάθεια για να γίνει ικανοποιητικά,
ενώ η αξία που θα προσέθετε στις περιπτώσεις χρήσης του \osv{} δεν πείθει για
την αναγκαιότητα του προς το παρόν.
