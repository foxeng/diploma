\chapter{Εισαγωγή}

\section{Κίνητρο}

\subsection{Γιατί \en{unikernel}}
Η τελευταία δεκαετία έχει χαρακτηριστεί από την κυριαρχία του μοντέλου του
υπολογιστικού νέφους (\en{cloud computing}) στην υπολογιστική υποδομή. Αυτό έχει
εξελιχθεί σε θεμέλιο λίθο σε πολλά πεδία εφαρμογών, κάνοντας πρακτικές
αρχιτεκτονικές που μέχρι πρότινος ήταν αδύνατες, για ένα ολοένα αυξανόμενο
πλήθος χρηστών. Επίσης, η ευρύτατη υιοθέτηση του έχει οδηγήσει σε διαχείριση
υπολογιστικών πόρων πρωτοφανούς κλίμακας. Η τεχνολογία που καθιστά δυνατό το
\en{cloud} στην παρούσα του μορφή είναι αυτή της εικονικοποίησης: της κατά
βούληση ``σμίλευσης'' πολλαπλών εικονικών μηχανών (\en{guests}) από ένα φυσικό
σύστημα (\host{}).

Παρά τις μεγάλες αλλαγές που έχει φέρει η έλευση του \en{cloud}, τα παραδοσιακά
κομμάτια της στοίβας λογισμικού, με κύριο εκπρόσωπο το λειτουργικό σύστημα,
έχουν επηρεαστεί ελάχιστα από αυτήν. Έτσι, σε αυτή την πλευρά κυριαρχούν τα
λειτουργικά συστήματα γενικού σκοπού όπως το \linux{}, με σχεδιαστικές επιλογές
και κληρονομιά δεκαετιών. Το γεγονός αυτό είναι αδιαμφισβήτητα ενδεικτικό της
υψηλής δυσκολίας που έχει η υλοποίηση τους, αλλά και της αξίας που έχουν τα
παρόντα συστήματα, κατόπιν συσσώρευσης μακράς αλυσίδας βελτιώσεων.

Ένας συνδυασμός παραγόντων όμως καθιστά πλέον εμφανές ότι τα γενικού σκοπού
λειτουργικά συστήματα δεν είναι η βέλτιστη λύση για τις ανάγκες των σύγχρονων
εφαρμογών. Σε αυτούς τους παράγοντες συγκαταλέγεται η μετάβαση από φυσικά σε
εικονικά μηχανήματα με την ακύρωση της υπόθεσης της αποκλειστικής χρήσης των
πόρων που αυτή συνεπάγεται. Επίσης, το ταχύ κλείσιμο του άλλοτε χάσματος ανάμεσα
στις επιδόσεις \en{I/O} και επεξεργασίας, που καθιστά σχεδόν απαγορευτική την
εμπλοκή του λειτουργικού συστήματος στο μονοπάτι δεδομένων (\en{data path}) μίας
εφαρμογής που απαιτεί ύψιστες επιδόσεις, εξ' ου και η δημοφιλία των \cite{dpdk,
spdk}. Τέλος, η προαναφερθείσα κλίμακα των υποδομών, που ενισχύει την ανάγκη για
βελτιστοποίηση της αποδοτικότητας, καθώς η υποβέλτιστη λειτουργία έχει μεγάλο
κόστος.

Τα \en{unikernels} είναι μία αξιόλογη πρόταση για αντικατάσταση των συμβατικών
λειτουργικών συστημάτων στους \en{guests}, όταν αυτά χρησιμοποιούνται για την
εκτέλεση μίας μόνο εφαρμογής. Προκύπτουν από τη συγχώνευση μίας εφαρμογής μαζί
με όσα υποστηρικτικά στοιχεία αυτή χρειάζεται (που τυπικά παρέχονται από το
λειτουργικό σύστημα), υπό τη μορφή βιβλιοθηκών, σε έναν κοινό χώρο διευθύνσεων.
Οι παραγόμενες εκτελέσιμες εικόνες εκτελούνται ως εικονικές μηχανές
επιτυγχάνοντας υψηλότερη αποδοτικότητα, αφού έχουν μικρότερο μέγεθος (οπότε
ταχύτερη μεταφορά και εκκίνηση \cite{jitsu} και χαμηλότερες απαιτήσεις
αποθηκευτικού χώρου), χαμηλότερες απαιτήσεις μνήμης και πιο αποδοτικές
λειτουργίες συστήματος λόγω πχ έλλειψης \en{mode switches} και απλουστευμένου
μοντέλου ασφάλειας.

\subsection{Γιατί κοινό σύστημα αρχείων και \viofs{}}
Τυπικά, τα συστήματα αρχείων είναι τοπικά, υλοποιημένα με συσκευές \en{block},
στο πλαίσιο του πυρήνα ενός λειτουργικού συστήματος. Υπάρχουν όμως περιπτώσεις
οι οποίες ωφελούνται από την κοινή χρήση του ίδιου συστήματος αρχείων από
περισσότερα συστήματα. Μία τέτοια περίπτωση είναι αυτή της εικονικοποίησης, όπου
\host{} και \guest{} μοιράζονται ένα σύστημα αρχείων (το οποίο συνήθως
προϋπάρχει στον \host{}). Ένα παράδειγμα όπου είναι σαφής η χρησιμότητα του
παραπάνω είναι στην εκκίνηση εικονικών μηχανών οι οποίες (ανα)διαμορφώνονται από
τον \host{}, ενώ ένα άλλο είναι η εκτέλεση βραχύβιων εικονικών μηχανών που
διαβάζουν δεδομένα εισόδου ή διαμόρφωσης και γράφουν δεδομένα εξόδου σε ένα
κοινόχρηστο σύστημα αρχείων.

Το \viofs{} είναι μια πρόσφατη προσπάθεια στο πεδίο των κοινόχρηστων συστημάτων
αρχείων, η πρώτη χτισμένη από την αρχή με αποκλειστικό στόχο τα περιβάλλοντα
εικονικοποίησης. Αυτή η εξειδίκευση του επιτρέπει να μην έχει καμία εξάρτηση
από δικτυακά πρωτόκολλα ή (εικονική) δικτυακή υποδομή, οδηγώντας αφενός σε
χαμηλότερες απαιτήσεις από τον \guest{} για τη χρήση του, αφετέρου σε καλύτερες
επιδόσεις και προσφορά σημασιολογίας τοπικού συστήματος αρχείων
\cite{virtiofs-website}. Σημαντικό ρόλο στην επίτευξη των προηγούμενων παίζει η
χρήση ενός χώρου κοινής μνήμης μεταξύ \host{} και \guest{} όπου απεικονίζονται
τα περιεχόμενα των αρχείων, το λεγόμενο \en{DAX (Direct Access) window}. Τα
παραπάνω το καθιστούν ιδανικό για χρήση σε ``\en{lightweight}'' \en{guests},
συμπεριλαμβανομένων των \en{unikernels}.

\section{Στόχοι}
Αυτή η εργασία έχει σκοπό την εξερεύνηση των δυνατοτήτων που προσφέρει το
\viofs{} σε ένα \en{unikernel}. Συγκεκριμένα, επιλέγουμε το \osv{} και
επεκτείνουμε την ήδη υπάρχουσα, στοιχειώδη υλοποίηση του για το \viofs{},
προσθέτοντας μεταξύ άλλων (\en{read-only}) υποστήριξη για το \en{DAX window}
καθώς και δυνατότητα εκκίνησης (\en{boot}) από ένα \viofs{} σύστημα αρχείων.
Έτσι καθίσταται πρακτική η χρήση του σε πολλές περιπτώσεις. Επίσης, αξιολογούμε
τις επιδόσεις της υλοποίησης μας σε σύγκριση με πλήθος άλλων συστημάτων αρχείων,
σε ένα εύρος αντιπροσωπευτικών σεναρίων.

Η εργασία όμως έχει και μη τεχνικούς στόχους που έχουν να κάνουν με το
μοντέλο του ελεύθερου λογισμικού / λογισμικού ανοιχτού κώδικα. Μιας και αμφότερα
τα έργα που εμπλέκονται (\osv{} και \viofs{}) είναι έργα ελεύθερου λογισμικού,
θεωρούμε ευκαιρία αλλά και υποχρέωση όλες οι επεκτάσεις μας στο \osv{} να
προσφερθούν για συμπερίληψη σε αυτό. Με αυτόν τον τρόπο και άλλοι χρήστες
του έργου μπορεί να επωφεληθούν από την εργασία μας, η οποία αποκτά αξία πέραν
της ακαδημαϊκής, ενώ, τέλος, οι κοινότητες των έργων αποκτούν ένα νέο, ενεργό
μέλος.

\section{Σχετική δουλειά}
Στα χρόνια από την ανακοίνωση του \en{MirageOS} \cite{mirageos} έχουν κάνει την
εμφάνιση τους πληθώρα \en{unikernel frameworks}, σχηματίζοντας ένα πολυποίκιλο
σύνολο. Μερικά, εκτός του \osv{}, είναι τα \en{RumpRun} \cite{rumprun},
\en{IncludeOS} \cite{includeos}, \en{HermitCore} \cite{hermitcore} και
\en{HermiTux} \cite{hermitux}, \en{ukl} \cite{ukl}, \en{Toro} \cite{toro} και
\en{Nanos} \cite{nanos}.

Η κυριότερη, προϋπάρχουσα εναλλακτική του \viofs{} είναι το \en{VirtFS}
\cite{virtfs} που βασίζεται στο δικτυακό πρωτόκολλο \en{9P} \cite{9p}. Το
\viofs{} είναι ακόμα ένα νεό έργο, υπό ενεργή ανάπτυξη, γι' αυτό και είναι
διαθέσιμες σχετικά λίγες υλοποιήσεις του. Από τη μεριά του \host{}, πέρα από την
κύρια υλοποίηση του στο \qemu{} \cite{virtiofs:qemu}, υπάρχει πλήρης υλοποίηση
του \viofs{} στον \en{cloud-hypervisor} \cite{cloud-hypervisor}, μία εναλλακτική
υλοποίηση του \en{virtiofsd} ονόματι \en{memfsd} από την κοινότητα των \en{nabla
containers} \cite{memfsd} καθώς και μία υλοποίηση στο \en{firecracker}
\cite{firecracker} η οποία έχει απορριφθεί προς το παρόν. Από τη μεριά του
\guest{}, πέρα από την κύρια υλοποίηση του στο \linux{} \cite{virtiofs:linux},
υπάρχει υλοποίηση στο \en{Toro} (ένα ακόμα \en{unikernel}, γραμμένο σε
\en{Pascal}) \cite{toro}, όπως και για \en{Windows} \cite{virtio-windows}.

\section{Οργάνωση του παρόντος}
Στο δεύτερο κεφάλαιο δίνεται μία πλήρης περιγραφή όλου του τεχνικού υποβάθρου
της εργασίας, από τις περιπτώσεις χρήσης του \en{cloud} έως και τις τεχνολογίες
με χρήση των οποίων υλοποιείται το \viofs{}. Το κεφάλαιο τρία αναλύει τη
διαδικασία και τα χαρακτηριστικά της υλοποίησης των επεκτάσεων μας στο \osv{}.
Το τέταρτο κεφάλαιο καταπιάνεται με την αξιολόγηση της τελευταίας, περιγράφοντας
τη μεθοδολογία που ακολουθήθηκε και εξετάζοντας τα αποτελέσματα, ενώ το πέμπτο
και τελευταίο κεφάλαιο περιέχει μία σύντομη ανασκόπηση και κατευθύνσεις για
περαιτέρω δουλειά στο μέλλον.
