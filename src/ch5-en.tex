\chapter{Conclusion and future work}

In the context of this thesis we successfully extended the implementation of
\viofs{} on \osv{} to use the DAX window (only for reading) and to boot with a
\viofs{} root file system. The former required the addition of a manager for the
DAX window, in order to achieve efficiency in the presence of multiple mappings.
In addition, we evaluated our implementation in different scenarios, concluding
that it significantly enhances \viofs{} performance, making it competitive to
that of \osv{}'s local file systems in the majority of cases and always superior
to that of NFS, the only other available shared file system on it. Finally, we
observed the value of a shared file system in practice, with rebuilding the
unikernel rendered redundant in most cases of modifications to the application.

The experience of using and developing on \osv{} proved very interesting, with
some great benefits as well as some points that can be improved. The main
advantage stems from the usage of C++, in combination with the nature of a
library OS and the significant work already put into the \osv{} kernel: often
during development one is under the impression of writing application code
instead of system code, with amenities like dynamic and semi-automated (through
smart pointers) memory management and standard libraries available. The unique
development workflow is also better compared to that of an application on a
general-purpose operating system, thanks to the short build and startup times.
Debugging is usually a challenge in unikernels, but in our experience, \osv{}
offers valuable support for it \cite{osv-wiki:debugging} and of course the
relative ease of programming aids in avoiding mistakes in the first place.
Documentation is an area with clear room for improvement, since very often the
most reliable way to figure out how something works is to read the relevant
source code, a likely discouraging factor for potential new users. This is of
course a common problem in software projects with very few contributors and
requires deliberate effort to change.

% TODO OPT: Unikernels in general (preference and forecast): ecosystem
% fragmentation, high-level language, ABI compatibility, target platform (light
% hypervisors), tooling

In the non-technical aspect of the work, we had the opportunity to contribute
useful additions to a free software project, also coming into contact with and
becoming members of the communities of two such projects, participating in
their processes and learning their workflows. This way we accomplished a primary
goal, that of giving back to the community of the project on which we relied
upon.

Both the communities we came into contact with, although different in size
and involvement of most of their members (small and purely voluntary in \osv{},
larger and professional in \viofs{}), had common core elements. Those were none
other than openness, welcoming, encouragement and support to new members in
addition to availability and willingness for collective development through
discussion, comments, well-intentioned critique and acknowledgement.
% TODO OPT: Point to discussions on the mailing lists?

As far as extensions of our work on \osv{} are concerned, one that is deemed
especially interesting is mapping the files in the application's memory via
\texttt{mmap()}. The challenge in this lies in essentially giving access to the
DAX window while maintaining \texttt{mmap()} semantics, thus eliminating the
copy made from the DAX window to the application's buffer necessitated by
\texttt{read()}. The greatest possible extension at this point is undoubtedly
write support, with or without the DAX window, to render the file system
read-write. It is estimated that this would require significant effort to
tackle adequately, while the value added in \osv{}'s common use cases is not
convincing of its necessity for now.
