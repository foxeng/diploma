\begin{otherlanguage}{english}
	\chapter*{\abstractname}
	\addcontentsline{toc}{chapter}{\abstractname}
	Cloud computing has been established as the dominant approach to compute
	infrastructure. The multitude of services it offers is grounded in
	virtualization technology. As cloud adoption grows worldwide, the expansion
	of its compute resources renders their efficient utilization by software
	imperative. One solution towards that is \emph{unikernels}, operating system
	kernels specialized to run a single instance of a single application at a
	time, sparing resources compared to a general-purpose kernel. As such,
	they have an advantage over general-purpose operating systems as far as
	resource efficiency is concerned, while also providing stronger isolation
	than containers, a more recent, popular alternative.

	Efficient while also secure virtualized guests' access to the underlying
	host's resources is a substantial challenge that comes with virtualization.
	In this aspect, \emph{virtio} has been a significant contribution, as a
	specification of paravirtual devices enabling efficient usage of the host's
	resources. For host-guest file sharing, until recently the solutions
	available relied on the virtual network connection between the two, with
	suboptimal results, since their colocation on the same physical machine was
	exploited. To address this issue, \emph{virtio-fs} has been proposed, as a
	virtio device offering guest access to a file system directory on the host,
	providing high performance and local file system semantics.

	This thesis is concerned with the implementation and evaluation of virtio-fs
	in a unikernel. We demonstrate that combining the two offers great benefits,
	both with regard to performance and the operational aspect in a cloud
	context. Moreover, we cater to the practical value of this work by
	contributing it to the open source project behind the unikernel we based it
	on.

	\section*{Keywords}
	\noindent
	virtualization, cloud, file system, unikernel, virtio, OSv, virtio-fs, QEMU
\end{otherlanguage}
