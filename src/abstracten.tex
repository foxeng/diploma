\begin{otherlanguage}{english}
	\chapter*{\abstractname}
	\addcontentsline{toc}{chapter}{\abstractname}
	Cloud computing is the dominant approach to compute infrastructure,
	established on the technology of virtualization. As the cloud expands,
	efficient utilization of its compute resources by software becomes
	imperative. One solution towards that are \emph{unikernels}, operating
	system kernels specialized to run a single application, sparing resources
	compared to a general-purpose kernel. Efficient access from virtualized
	guests to the underlying host's resources is a substantial challenge in
	virtualization. In this aspect, \emph{virtio} has been a significant
	contribution, as a specification of paravirtual devices enabling efficient
	usage of the host's resources. For host-guest file sharing, \emph{\viofs{}}
	has been proposed, as a virtio device offering guest access to a file system
	directory on the host, providing high performance and local file system
	semantics.

	This thesis is concerned with the implementation and evaluation of \viofs{}
	in the context of the \osv unikernel. We demonstrate that combining the two
	offers great benefits, both with regard to performance achieved, which
	is comparable to local file systems, and the operational aspect in a cloud
	context. Moreover, the above are carried out fully within the open-source
	project behind the unikernel we based our work on. This way, the resulting
	product gains practical value, being a useful contribution to the project,
	thus achieving a pivotal, non-technical goal. Furthermore, we explore how
	open-source software projects and the communities around them work, as we
	become active members of one.

	\section*{Keywords}
	\noindent
	virtualization, cloud, file system, unikernel, virtio, \osv{}, \viofs{},
	\qemu{}
\end{otherlanguage}
