\chapter*{\abstractname}
\addcontentsline{toc}{chapter}{\abstractname}

Το \en{cloud computing} είναι πλέον εδραιωμένο ως η κυρίαρχη προσέγγιση προς την
υπολογιστική υποδομή. Η πληθώρα των υπηρεσιών που παρέχονται από αυτήν την
υποδομή έχει ως θεμέλιο λίθο την τεχνολογία της εικονικοποίησης
(\en{virtualization}). Καθώς το \en{cloud} υιοθετείται όλο και περισσότερο σε
παγκόσμια κλίμακα, το αυξανόμενο μέγεθος των υπολογιστικών πόρων καθιστά
ολοένα και πιο επιτακτική την αποδοτική χρήση αυτών από το λογισμικό. Προς αυτό
τον σκοπό, μια λύση είναι τα \emph{\en{unikernels}}, πυρήνες λειτουργικών
συστημάτων εξειδικευμένοι για να τρέχουν ένα στιγμιότυπο μίας μόνο εφαρμογής
κάθε φορά, εξοικονομώντας πόρους σε σχέση με έναν γενικού σκοπού πυρήνα. Έτσι,
πλεονεκτούν έναντι στα γενικού σκοπού λειτουργικά συστήματα ως προς την
αποδοτικότητα, ενώ ταυτόχρονα προσφέρουν πιο ισχυρή απομόνωση από τα
\en{containers}, τα οποία αποτελούν μια πιο πρόσφατη, δημοφιλή εναλλακτική.

Η αποδοτική και ταυτόχρονα ασφαλής πρόσβαση των \en{virtualized guests} στους
πόρους του \en{host} συστήματος είναι μια μεγάλη πρόκληση που συνοδεύει την
εικονικοποίηση. Σε αυτόν τον τομέα, σημαντική συνεισφορά αποτελεί το
\emph{\en{virtio}}, μια προδιαγραφή για \en{paravirtualized} συσκευές που
καθιστά εφικτή την αποδοτική χρήση των πόρων του \en{host}. Για την κοινοχρησία
αρχείων ανάμεσα σε \en{host} και \en{guest} μέχρι πρόσφατα οι διαθέσιμες λύσεις
βασίζονταν στην εικονική σύνδεση δικτύου μεταξύ των δύο, με υποβέλτιστα
αποτελέσματα αφού δεν εκμεταλλεύονταν το γεγονός της συνύπαρξης τους στο ίδιο
φυσικό μηχάνημα. Προκειμένου να επιλύσει αυτό το ζήτημα έχει προταθεί το
\emph{\en{virtio-fs}}, μια εικονική \en{virtio} συσκευή που προσφέρει πρόσβαση
σε έναν κατάλογο του συστήματος αρχείων του \en{host} από τον \en{guest},
παρέχοντας υψηλές επιδόσεις και σημασιολογία τοπικού συστήματος αρχείων.

Αντικείμενο της παρούσας διπλωματικής εργασίας είναι η υλοποίηση και αξιολόγηση
της χρήσης του \en{virtio-fs} σε ένα \en{unikernel}. Δείχνουμε ότι ο συνδυασμός
αυτών των δύο έχει σημαντικά πλεονεκτήματα, τόσο από άποψη επιδόσεων, όσο και
από διαχειριστική άποψη στο πλαίσιο του \en{cloud}. Επίσης, φροντίζουμε ώστε το
προϊόν της εργασίας να έχει πρακτική αξία, συνεισφέροντας το στο έργο ανοιχτού
λογισμικού του \en{unikernel} στο οποίο βασιστήκαμε.

\section*{Λέξεις κλειδιά}
\noindent
εικονικοποίηση, νέφος, σύστημα αρχείων, \en{unikernel}, \en{virtio}, \en{OSv},
\en{virtio-fs}, \en{QEMU}
